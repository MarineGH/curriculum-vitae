% DOCUMENTATION
% https://fr.wikibooks.org/wiki/LaTeX

\documentclass[a4paper, 11pt]{book}
\usepackage[utf8]{inputenc}
\usepackage{graphicx}
% \usepackage[frenchb]{babel}
\begin{document}

\title{Une histoire de l'informatique}
\author{Alexandre Bouvier}
\date{}

\mainmatter

\maketitle

\chapter*{ Introduction }

\section*{ La Logique à l'origine des ordinateurs }

\paragraph{
  La recherche de la vérité a toujours été à la base des Sciences. Et dans une démonstration scientifique, la preuve logique est la voie royale d'accès à la vérité. Apparu chez les philosophes grecs, elle a été enrichi par l'anglais, Georges Boole qui lui a donné ses lettres de noblesse. De 1844 à 1854, il crée une algèbre binaire, n'acceptant que deux valeurs numériques : 0 et 1. on appellera cette logique booléenne. Il veut affranchir la logique de la réduction aux mathématiques à laquelle elle était étroitement lié et veut en faire une discipline à part entière en mettant sur pied les notions de quantificateur (ET, OU, QUELQUE SOIT, IL EXISTE, IMPLICATION, EQUIVALENCE). Par la suite, les logiciens voulaient que le nouveau langage logique donne aux mathématiques de solides fondements (nottamment via la Théorie des Ensembles). La logique, pour Hilbert, est une image de la plus haute forme de vérité. Selon lui "les Mathématiques ne doivent plus rien admettre  qui soit intuitivement évident" comme par exemple deux droites parallèles ne se croisent jamais.
}

\paragraph{
  Plusieurs paradoxe mettent en avant l'illogisme du langage courant et les mathématiciens ne veulent pas tomber dans les mêmes travers. Mais cette recherche de la vérité en Mathématique basé sur la logique butte sur plusieurs problèmes. En 1900 à l'occasion d'un congrès international de mathématiciens tenu à Paris, David Hilbert, le mathématicien allemand, propose sa fameuse liste des 23 problèmes que les Mathématiques moderne n'avaient pas résolus. Même au XXIe siècle, elle est considérée comme étant la compilation de problèmes ayant eu le plus d'influence en mathématiques.
}

\paragraph{
  Malheureusement dans les sciences en générale et dans les Mathématiques en particulier cette recherche de la vérité absolue n'a jamais abouti. Elle a même rendu fou de nombreux savants. En 1920 Godel siffle la fin de la récréation avec son théorème d'incomplétude.
}

\section*{ Théorème d'incomplétude de Godel }

\paragraph{
  Ce théorème affirme qu'en Mathématiques, il existerait toujours des vérités qui sont indémontrables. Von Neumann dit que c'est terminé en entendant le théorème d'incomplétude de Gödel. Alan Turing répond "OK on peut pas tout prouver! Donc voyons ce qu'on peut prouver".
}

\section*{ Alan Turing }

\paragraph{Alan Turing est alors tout jeune chercheur quand Godel expose son théorème d'incomplétude. La logique qui est la voie royale des sciences ne séduit plus. La mode est à la résolution des problèmes d'Hilbert. Le cambrdigien s'arrete sur le 11ème. En 1936, la publication d'un article de logique mathématique On computable Numbers, with an Application to the Entscheidungsproblem \footnote{Référence de la publication} constitue avec d'autres recherches fondamentales un cadre théorique qui intéressera plus tard les fondateurs de la "science informatique". La machine de Turing est une abstraction modélisant un "être calculant" pour démontrer une proposition de logique pure. À l'époque n'a rien à voir avec un projet de machine.}

% \paragraph{Tous les ordinateurs et microprocesseurs d'aujourd'hui sont encore basés sur le principe énoncé par Turing. Son aspect élémentaire et basique sert également à déterminer si un calcul peut être réalisé sous forme automatique (algorithmique).}

\chapter{ Les Racines }

\chapter{ 1936 }

\chapter{ WWII }

\chapter{ 1945 }

\chapter{ 1950 }

\chapter{ 1960 }

\chapter{ 1970 }

\chapter{ 1980 : L'explosion de l'ordinateur personnel }

\paragraph{
  Dans les années 80, plusieurs innovations techniques vont jouer une rôle majeur pour démocratiser les ordinateurs et passer du computer au personnal computer.
}

\paragraph{
  D'abord, les composants des ordinateurs (processeurs, etc.) deviennent de plus en plus puissant année après année. Il traite toujours plus de calculs avec un prix de moins en moins cher.
}

\section*{Les insolubles problèmes de compatibilité}

\paragraph{
  C'est en 1981 qu'IBM lance son premier ordinateur personnel, l'IBM PC. Il fonctionne avec un terminal classique, c'est à dire uniquement avec un invite de commande. Apple, la société de Steve Jobs et Steve Wozniak lui emboîte le pas avec l'Apple I et le Lisa deux ans plus tard. La limite de ces ordinateurs est l'incompatbilité des leurs logiciels. Il est fréquent que les programmes ne soient pas compatibles même entre ordinateurs d'une même marque. C'est à dire qu'un logiciel écrit pour l'Apple I ne fonctionnera pas pour le Lisa. L'ergonomie est très primaire. Tout se fait par des lignes de commande qui se lance sur un terminal tout de noir vêtu.
}

\section*{Microsoft}

\paragraph{
  Comme vendeur de logiciels le plus gors est à l'époque Microsoft. La fime de Seattle avait écrit quelques application pour l'Apple II dont le tableau Multiplan, futur Excel.
}

\paragraph{
  Lancé en 1975, la société Microsoft est une des premières à s'être spécialisée dans les logiciels. Gates avait fondé Microsoft pour finaliser la vente de leur Basic pour l'Altair. Basic est un langage de programmation conçu à l'origine pour permettre à des néophytes en informatique d'écrire des programmes pouvant tourner sur divers ordinateurs. A l'époque le système d'exploitation (OS) pour les IBM est le MS-DOS (PC-DOS). Il fonctionne avec des invites de commandes. Microsoft s'est réservé le droit de commercialiser sa propre version du MS-DOS à des ordinateurs non-IBM. IBM accepte en échange d'un ristourne sur le prix de la licence MS-DOS. Il ne le sait pas encore mais IBM vient de signer son propre arrêt de mort sur le marché des PC.
}

\paragraph{
  En effet à part Apple qui choisit de construire son propre OS, les nouvelles sociétés de Hardware vont utiliser Microsoft. Et une va particulièrement tirer son épingle du jeu: Compaq.
}

\section*{Un nouvel arrivant: Compaq}

\paragraph{
  Dans les années 80 après la sortie des premiers ordinateurs personnels destinés aux grand public, Compaq invente l'ordinateur portable. La compagnie originaire de Houston sort un ordinateur de la taille d'une malette équipé d'une poignée. Créé au début des années 1980, la société Texanne a été parmi les premières à prendre au sérieux les problèmes de compatibilité. Pour concurencer IBM qui regnait en maître sur le marché, Compaq a mis en place un compatibilité entre son ordinateur et certains logiciels tournant sur IBM.
}

\paragraph{
  Par exemple à l'époque, une société qui achetait 100 ordinateurs Apple II puis faisait développer des logiciels, en général, par un société externe. Si ensuite elle souhaitait s'aggrandir et acheter des nouveaux ordinateurs d'une autre marque, tout le software était à refaire. En faisant du *retro-ingeniering* Compaq vendait des PC compatibles. Mais seulement avec des IBM et pas des Apple. Devant le succès commercial du Compaq, IBM décida de sortir lui aussi son portable (lequel??). Mais les *softwares* des autres IBM n'étaient pas comptabibles avec le PC portable d'IBM. En résumé si vous aviez développé des logiciels sur un IBM puis que vous souhaitiez renouvellliez votre stock de vieux PC, les logiciels étaient bon pour la poubelle.
}

\section*{Le MacIntosh}

\paragraph{
  En 1983 Apple sort le Lisa. C'est la premier ordinateur avec une interface graphique. Son échec commercial fut cuisant mais Apple réitère en 1984 avec le MacIntosh. Steve Jobs est convaincu de la pertinence d'une interface graphique pour toucher un public plus large que les spécialistes et équiper tous les foyers américains d'un ordinateur personnel. C'est au cours d'une visite dans les locaux de XeroX que Steve Jobs trouve l'inspiration pour ses deux innovations majeurs.
}

\paragraph{
  La première innovation est la fenêtre graphique (*window*) en tant que répresentation visuelle d'un dossier (*directory*). La deuxième est la souris qui permettait une utilisation facile des fichiers. Les deux ne sont pas breveté par XeroX. Steve Jobs est le premier a avoir l'idée d'utilisation la souris pour la manipulation des fichiers (*files*) et dossiers (*directory*).
}

\paragraph{
  Le premier MacIntosh (1984) était né. C'est le premier PC d'Apple même s'il n'est pas encore transportable. C'est le premier PC de l'histoire avec une souris et des fenêtres de navigation sur le bureau. Le MacInstosh était très lent. Il n'avait que 128kO de RAM et un lecteur de disquette faisant office de disque dur. C'est à dire pas de mémoire. Les logiciels n'étaient pas compatibles. Il reprenait plusieurs caractéristiques du Lisa, comme le processeur Motorola 68000, mais pour un prix bien plus abordable 2 500 dollars. Ils ont pu obtenir ce prix bas grâce à l'abandon de quelques fonctionnalités comme le multitâche. Plusieurs applications utilisait la souris comme MacPaint et MacWrite.
}

\paragraph{
  Malheureusement Steve Jobs ne survivra pas à l'échec du Lisa puisqu'il fut viré de l'entreprise en septembre 1985. Apple ne sortira plus rien d'innovant après cela. Un retard fatal, l'industrie va avancer sans elle et Microsoft va imposer ses normes. Apple se retrouva hors-jeu pour un long moment.
}

\paragraph{
  C'est le 27 septembre 1983 que commence le projet GNU. Le projet fondé par Richard Stallman a l'ambition de créer des software non protégés pas des brevets et surtout compatible. La compatiblité pouvant être vu comme un implication d'un software libre. En effet si le code est libre il est plus facile pour les créateurs de hardware de fabriquer une machine qui lit le logiciel. En plus de logiciels GNU veut aussi sortir un OS pouvant tourner sur plusieurs machines. C'est bien après la bataille, seulement en 1997, que GNU sort une interface graphique pour son OS.
}

\section*{1985: Windows lance sur *Operating System*}

\paragraph{
  En 1985, Windows sort son premier OS Windows 1.0. Une version médiocre et un échec commercial dû en partie à sa mauvaises gestion des fenêtres. Elles n'avaient pas de bouton de fermeture et elle ne se chevauchait pas. Le nouveau CEO d'Apple, Sculley, voulait poursuivre en justice Microsoft pour avoir copier le système des fenêtres du MacIntosh. Mais Microsoft menaca de bloquer les ventes de logiciels pour Mac (Word et Excel). À l'époque, Bill Gates déclare: "Xerox était notre riche voisin à tous les deux, et que je suis entré chez lui pour lui voler sa télévision, j'ai découvert que [Steve Jobs] l'avait déjà emporté".
}

\section*{La parenthèse NeXt}

\paragraph{
  En 1988, Steve Jobs fonde NeXT, une société d'ordinateur pour les centres de recherches. Le système d'exploitation s'appelait NeXTStep. IBM l'utilise un temps pour mettre en concurrence Windows. Compaq et Dell voulait aussi l'iutiliser. Steve Jobs investit dans Pixar, le département informatique de LucasFilm. Elle devient alors une société indépendante. En déconfiture à l'époque, Pixar gérait le graphisme et les animations 3D que Steve vouait intégrer au NeXt. En 1990, NeXT arrête de fabriquer des ordinateurs pour se concentrer sur l'OS.
}

\paragraph{
  Dans son autobiographie, Richard Stallman analyse ce changement de paradigme sur le marché informatique: "Les logiciels autrefois considéré comme une forme de garniture offerte par les fabricants pour donner plus saveurs à leurs coûteux systèmes informatiques, deviennent rapidement le plat principal."
}

\paragraph{
  La fin des années 80 marque la prise de pouvoir du software sur le hardware, c'est à dire des logiciels sur la machine. Et justement, le logiciel qui va devenir le plat principal est en train de naître.
}

\section*{1989: La naissance du World Wide Web}

\paragraph{
  C'est en 1989 que le chercheur britannique Tim Berners-Lee invente le World Wide Web sur une machine NeXT. Il travaille au CERN un laboratoire de physique des particules.
}

\paragraph{
  *À l’origine, le projet, baptisé « World Wide Web », a été conçu et développé pour que des scientifiques travaillant dans des universités et instituts du monde entier puissent s'échanger des informations instantanément.*
}

\paragraph{
  *L'idée de base du WWW était de combiner les technologies des ordinateurs personnels, des réseaux informatiques et de l'hypertexte pour créer un système d'information mondial, puissant et facile à utiliser.* par la communauté des chercheurs (site web du CERN)
}

\paragraph{
  *Le premier site Web créé au CERN, et dans le monde, était destiné au projet World Wide Web lui-même. Il était hébergé sur l’ordinateur NeXT de Tim Berners-Lee. En 2013, le CERN a entrepris de remettre en service ce premier site Web : info.cern.ch.* qui est aujourd'hui le site web du CERN. (site web du CERN)
}

\paragraph{
  Pour mieux comprendre ce que cela signifie, il faut s'intéresser au fonctionnement d'Internet: pour voir une page web, l'ordinateur reçoit un petit paquet de données pour l'image, un autre pour le texte, etc. Ces paquets de données voyagent des serveurs jusqu'à votre PC grâce à une infrastructure de câbles et de fibres, un peu comme des camions voyagent de Paris à Toulouse en passant par des autoroutes. C'est ce qu'on appelle l'internet. Sauf que pour faire cette route, il faut s'acquitter d'un droit de passage à l'entreprise qui possède les infrastructures. Sur Internet, chacun paie donc à des opérateurs de télécommunications (en France Orange, Bouygues, Free...) une certaine somme pour accéder à leur infrastructure, un peu comme un camion paie Vinci pour son trajet d'autoroute.
}

\paragraph{
  *Tim Berners-Lee écrit la première proposition de création du World Wide Web en mars 1989 et sa seconde proposition en mai 1990. Puis, en novembre 1990, l’ingénieur en systèmes belge Robert Cailliau le rejoint et ils élaborent ensemble une proposition formelle pour un système de gestion de l'information esquissant les concepts fondamentaux et définissant les principaux termes liés au Web. Le document décrit un « projet hypertexte » appelé WorldWideWeb, dans lequel un « web » (une toile) de « documents hypertextes » peut être vu par des « navigateurs ».*
}

\paragraph{
  *Fin 1990, Tim Berners-Lee rend opérationnel le premier serveur et navigateur Web au CERN, concrétisant ainsi ses idées. Il avait développé le code pour le premier serveur Web sur un ordinateur NeXT. Pour éviter qu'on ne l'éteigne accidentellement, une étiquette avait été collée sur l'ordinateur, où il était écrit à la main, en rouge : « Cette machine est un serveur. NE PAS ÉTEINDRE !! »*
}

\paragraph{
  *Info.cern.ch était l’adresse du tout premier site et serveur Web, qui était hébergé sur un ordinateur NeXT du CERN. L’adresse de la première page Web était http://info.cern.ch/hypertext/WWW/TheProject.html*
}

\paragraph{
  *La page comportait essentiellement des informations relatives au projet WWW, notamment une description de ce qu'est l'hypertexte, des détails techniques pour la création d'un serveur Web, et des liens vers d'autres serveurs, qui étaient ajoutés au fur et à mesure qu'ils devenaient disponibles.*
}

\paragraph{
  *La conception du WWW permettait d'avoir facilement accès à l'information existante ; une page Web rudimentaire proposait des liens utiles pour les scientifiques du CERN (par exemple l'annuaire du CERN, ou encore des guides d'utilisation des ordinateurs centraux du CERN). La recherche se faisait par mots-clés, car il n'y avait pas encore de moteurs de recherche.*
}

\paragraph{
  Le premier navigateur de Tim Berners-Lee fonctionnait uniquement sur NeXT. Le chercheur conçoit rapidement un navigateur pouvant être exécuté sur d'autres ordinateurs que des NeXT. Mais le navigateur était très rudimentaire et peu convivial même s'il permettait déjà de modifier les pages directement depuis le navigateur.
}

\paragraph{
  *En 1991, Tim Berners-Lee lance son premier logiciel WWW, qui incluait le navigateur en mode ligne, un logiciel pour le serveur Web et une bibliothèque pour les développeurs. En mars de cette même année, le logiciel devient accessible à d'autres collègues sur des ordinateurs du CERN.*
}

\paragraph{
  *Grâce aux efforts de Paul Kunz et Louise Addis, le premier serveur Web aux États-Unis est mis en ligne en décembre 1991, là aussi dans un laboratoire de physique des particules : le Centre de l'accélérateur linéaire de Stanford (SLAC), en Californie. Il n'y avait alors pour ainsi dire que deux sortes de navigateur. L'un était la version qui avait servi au développement initial, sophistiquée mais disponible uniquement sur des machines NeXT. L'autre était le navigateur en mode ligne, très simple à installer et à exécuter sur n'importe quelle plateforme, mais limité en puissance et peu convivial. Il était évident que la petite équipe du CERN à l'origine du système ne pourrait à elle seule effectuer le travail nécessaire pour le développer. Aussi Tim Berners-Lee lance-t-il un appel via l'Internet pour que d'autres développeurs viennent leur prêter main-forte. Plusieurs personnes créent alors des navigateurs, la plupart exécutables dans l'environnement X-Window. Les plus connus de cette époque sont MIDAS, de Tony Johnson (qui travaille alors auprès du SLAC), Viola, de Pei Wei (qui travaillait pour l'éditeur d'ouvrages techniques O'Reilly), et Erwise, d'un groupe d'étudiants finlandais de l'Université de technologie d'Helsinki.* De plus Emacs se développe aussi pour lire des pages web.
}


\chapter{ 1990 }

\tableofcontents

\end{document}

