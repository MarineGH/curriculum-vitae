\documentclass[a4paper, 11pt]{book}
\usepackage[utf8]{inputenc}
\usepackage{graphicx}
% \usepackage[frenchb]{babel}
\begin{document}

\title{Une histoire de l'informatique}
\author{Alexandre Bouvier}
\date{}

\maketitle

\chapter*{Introduction}

\section{La Logique à l'origine des ordinateurs}

\paragraph{
  La recherche de la vérité a toujours été à la base des Sciences. Et dans une démonstration scientifique, la preuve logique est la voie royale d'accès à la vérité. Apparu chez les philosophes grecs, elle a été enrichi par l'anglais, Georges Boole qui lui a donné ses lettres de noblesse. De 1844 à 1854, il crée une algèbre binaire, n'acceptant que deux valeurs numériques : 0 et 1. on appellera cette logique booléenne. Il veut affranchir la logique de la réduction aux mathématiques à laquelle elle était étroitement lié et veut en faire une discipline à part entière en mettant sur pied les notions de quantificateur (ET, OU, QUELQUE SOIT, IL EXISTE, IMPLICATION, EQUIVALENCE). Par la suite, les logiciens voulaient que le nouveau langage logique donne aux mathématiques de solides fondements (nottamment via la Théorie des Ensembles). La logique, pour Hilbert, est une image de la plus haute forme de vérité. Selon lui "les Mathématiques ne doivent plus rien admettre  qui soit intuitivement évident" comme par exemple deux droites parallèles ne se croisent jamais.
}

\paragraph{
  Plusieurs paradoxe mettent en avant l'illogisme du langage courant et les mathématiciens ne veulent pas tomber dans les mêmes travers. Mais cette recherche de la vérité en Mathématique basé sur la logique butte sur plusieurs problèmes. En 1900 à l'occasion d'un congrès international de mathématiciens tenu à Paris, David Hilbert, le mathématicien allemand, propose sa fameuse liste des 23 problèmes que les mathématiques moderne n'avaient pas résolus. Même au XXIe siècle, elle est considérée comme étant la compilation de problèmes ayant eu le plus d'influence en mathématiques.
}

\paragraph{
  Malheureusement dans les sciences en générale et dans les Mathématiques en particulier cette recherche de la vérité absolue n'a jamais abouti. Elle a même rendu fou de nombreux savants. En 1920 Godel siffle la fin de la récréation avec son théorème d'incomplétude.

}

\section{Théorème d'incomplétude de Godel}

\paragraph{
  Ce théorème affirme qu'en Mathématique, il existerait toujours des vérités qui sont indémontrables.
  Von Neumann dit que c'est terminé en entendant le théorème d'incomplétude de Gödel. Alan Turing répond "OK on peut pas tout prouver! Donc voyons ce qu'on peut prouver".
}

\section{Alan Turing}

\paragraph{
  Alan Turing est alors tout jeune chercheur quand Godel expose son théorème d'incomplétude. La logique qui est la voie royale des sciences ne séduit plus. La mode est à la résolution des problèmes d'Hilbert. Le cambrdigien s'arrete sur le 11ème. En 1936, la publication d'un article de logique mathématique *On computable Numbers, with an Application to the Entscheidungsproblem* constitue avec d'autres recherches fondamentales un cadre théorique qui intéressera plus tard les fondateurs de la "science informatique". La machine de Turing est une abstraction modélisant un "être calculant" pour démontrer une proposition de logique pure. À l'époque n'a rien à voir avec un projet de machine.
}

\paragraph{
  Tous les ordinateurs et microprocesseurs d'aujourd'hui sont encore basés sur le principe énoncé par Turing. Son aspect élémentaire et basique sert également à déterminer si un calcul peut être réalisé sous forme automatique (algorithmique).
}

\chapter{Les Racines}

  Hello World

\chapter{1936}

\tableofcontents

\end{document}

