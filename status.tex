\documentclass[a4paper, 11pt]{article}
\usepackage[utf8]{inputenc}
\usepackage[T1]{fontenc}
\usepackage{lmodern}
\usepackage{graphicx}
% \usepackage[french]{babel}

\begin{document}

\title{« KORIUM »}
\author{
  Société par actions simplifiée au capital de 3.000 euros\\
  Siège social : 52, Rue de Courcelles\\
  75008 PARIS\\
}
\date{}

\maketitle

\pagebreak

\paragraph{
  TITRE I\\
  FORME - DÉNOMINATION - OBJET - SIEGE - DURÉE
}

\section*{ARTICLE 1 - FORME}

\subparagraph{
  La Société est une Société par actions simplifiée régie par les dispositions  légales applicables et par les présents Statuts.
}

\subparagraph{
  Elle fonctionne indifféremment sous la même forme avec un ou plusieurs associés.
}

\subparagraph{
  Elle peut émettre toutes valeurs mobilières définies à l'article L 211-2 du Code monétaire et financier, donnant accès au capital ou de titres de créances, dans les conditions prévues par la loi et les présents statuts.
}

\section*{ARTICLE 2 - DÉNOMINATION}

\subparagraph{
  La dénomination sociale est « KORIUM »
}

\subparagraph{
  Dans tous les actes et documents émanant de la Société et destinés aux tiers, la dénomination doit être précédée ou suivie immédiatement des mots "Société par Actions Simplifiée" ou des initiales "S.A.S." et de l'énonciation du montant du capital social.
}

\section*{ARTICLE 3 - OBJET}

\subparagraph{
  La Société a pour Objet, directement ou indirectement, en FRANCE ou hors de FRANCE :
}

\subparagraph{
  - La vente, la distribution, IC négoce par tous moyens, tous supports et sous toutes formes en ce compris la vente en gros, demi-gros, au détail, au travers d'un réseau de distribution, par vente directe, que ce soit cn magasin ou par voie dématérialisée de tous produits fabriqués en cuir et/ ou en toutes autres matières et ce, en tout en partie.\\
  - Toutes prestations de services se rapportant à cette activité mais plus encore le Conseil sur l'organisation de tous types d'entreprise que ce soit dans le cadre de restructuration, de projet de développement, mais aussi l'acquisition, la conquêtc de parts de marché par des entreprises et en particulier tes structures artisanales,
}

\subparagraph{
  Et généralement toutes opérations financières, commerciales, industrielles, mobilières ou immobilières pour son compte ou pour celui d'autrui pouvant se rattacher directement ou indirectement aux Objets ci-dessus et susceptibles faciliter la réalisation ou le développement.
}
\section*{ARTICLE 4 - SIEGE SOCIAL}

\subparagraph{
  Le siège de la Société est sis :\\
  52, Rue de Courcelles\\
  75008 PARIS
}

\subparagraph{
  Il peut être transféré par décision du Président qui est habilité à modifier les statuts en conséquence.
}

\section*{ARTICLE 5 - DURÉE}

\subparagraph{
  La durée de la Société est de 99 années compter de son immatriculation au Registre du Commerce et des Sociétés, snuf les cas de prorogation ou dc dissolution anticipée.
}

\paragraph{
  Titre II\\
  APPORTS - CAPITAL - ACTIONS
}

\section*{ARTICLE 6 - APPORTS}

\section*{ARTICLE 7 - CAPITAL SOCIAL}

\subparagraph{
  Le capital social est fixé à la TROIS MILLE EUROS (3 000 EUROS).
}

\subparagraph{
  Il est divisé en trois cents (300) actions de dix (1O) euros chacune, souscrites en totalité et libérées en intégralité, les actions sont toutes de même catégorie.
}

\section*{ARTICLE 8 - AUGMENTATION DU CAPITAL SOCIAL}

\subparagraph{
  8-1 Le capital ne peut être augmenté que par une décision collective des associés statuant sur le rapport du Président.
}

\subparagraph{
  Le capital social peut être augmenté soit par émission chictions ordinaires ou de préférence, soit par majoration du montant nominal des titres de capital existants. Il peut également être augmenté par l'exercice des droits attachés des valeurs mobilières donnant accès au capital, dans les conditions prévues par la loi.
}

\subparagraph{
  Les titres de capital nouveaux sont émis soit à leur montant nominal, soit à ce montant majoré prime d'émission. Ils sont libérés soit par apport en numéraire y compris par compensation avec des créances liquides el exigibles sur la Société, soit par apport en nature, soit par incorporation de réserves, bénéfices ou primes d'émission, soit en conséquence d'une fusion ou d'une scission. Ils peuvent aussi être libérés consécutivement à l'exercice d'un droit attaché à des valeurs mobilières donnant accès au capital comprenant, le cas échéant, le versernent des sommes
}

\subparagraph{
  8-2 Les associés peuvent déléguer au Président les pouvoirs nécessaires à l'effet de réaliser ou de décider, dans les conditions et délais prévus par la loi, l'augmentation du capital.
}

\subparagraph{
  8-3 En cas d'augmentation du capital en numéraire ou d'émission de valeurs mobilières donnant accès au capital ou donnant droit à l'attribution de titres de créances, les assa:iés ont, proportionnellement au montant de leurs actions, un droit de préférence à la souscription des nouveaux titres émis. Toutefois, les associés peuvent renoncer à titre individuel à leur droit préférentiel de souscription et la décision d'augmentation du capital peut supprimer ce droit préférentiel dans les conditions prévues par la loi.
}

\section*{ARTICLE 9 - LIBERATION DU CAPITAL}

\subparagraph{
  9-1 Toute souscription d'actions en numéraire est obligatoirement accompagnée du versement de la quantité minimale prévue par la loi et, le cas échéant, de la totalité de la prime d'émission. Le surplus est payable en une ou plusieurs fois aux époques et dans les proportions qui seront fixées par le Président en conformité de la loi.
}

\subparagraph{
  Les appels de fonds sont portés à la connaissance des associés quinze jours au moins avant l'époque fixée pour chaque versement, par lettres recommandées avec demande de réception. Les associés ont la faculté d'effectuer des versements anticipés.
}

\subparagraph{
  9-2 A défaut de libération des actions l'expiration du délai fixé par le Président, les sommes exigibles sont, de plein droit, productives d'intérêt au taux de l'intérêt légal, à partir de la date d'exigibilité. le tout sans préjudice des recours et sanctions prévus par la loi.
}

\section*{ARTICLE 10 - REDUCTION DU CAPITAL, SOCIAL}

\subparagraph{
  La réduction du capital est autorisée Oti décidée par l'Assemblée Générale Extraordinaire qui peut déléguer au Président tous pouvoirs pour la réaliser, En aucun cas, elle ne peut porter atteinte l'égalité des associés.
}

\subparagraph{
  La réduction du Capital un montant inférieur au minimum légal ne peut être décidée sous la condition suspensive dune augmentation de capital destinée à amener celui-ci à un montant au moins égal à ce montant minimum, sauf transformation de la Société en Société d'une autre forme.
}

\section*{ARTICLE 11 - FORME DES ACTIONS}

\subparagraph{
  Les actions sont obligatoirement nominatives. Elles donnent lieu une inscription compte individuel dans les conditions Ct selon les modalités prévues par les dispositions législatives et réglementaires en vigueur.
}

\subparagraph{
  Ces comptes individuels peuvent être des comptes nominatifs ou des comptes nominatifs administrés au choix dc l'associé concerné. 'l'out associé peut demander à la Société la délivrance attestation d'inscription en compte.
}

\section*{ARTICLE 12 - INDIVISIBILITÉ DES ACTIONS}

\subparagraph{
  Les actions sont indivisibles à l'égard de la Société. Les copropriétaires indivis d'actions sont représentés aux Assemblées Générales par l'un d'eux ou par un mandataire commun de leur choix.  À défaut d'accord entre eux sur le choix d'un mandataire, celui-ci est désigné par Ordonnance du Président du Tribunal de Commerce statuant en référé à la demande du copropriétaire le plus diligent.
}

\subparagraph{
  Le droit de vote attaché à l'action appartient l'usufruitier dans les Assemblées Générales Ordinaires et au nu-propriétaire dans les Assemblées Générales Extraordinaires. Cependant, les associés peuvent convenir entre eux de toute autre répartition pour l'exercice du droit de vote aux Assemblées Générales. En Ce cas, ils devront porter leur convention à la connaissance de la Société par lettre recommandée adressée au Siège social, la Société étant tenue dc respecter cette pour toute Assemblée Générale qui se réunirait après l'expiration d'un délai d'un mois suivant de la lettte recommandée, le cachet de la poste faisant foi de la date dexpédition.  Nonobstant les dispositions ci-dessus, le nu-propriétaire a le droit dc participer toutes les assemblées générales.
}

\subparagraph{
  Le droit de l'associé dobtenir communication de documents sociaux ou de les consulter peut également être exercé par chacun des copropriétaires d'actions indivises, par l'usufruitier et le nu- propriétaire d'actions.
}

\section*{ARTICLE 13 - CESSION ET TRANSMISSION DES ACTIONS}

\subparagraph{
  13.1 La propriété des actions résulte de leur inscription en compte individuel au nom du Ou des titulaires sur les registres tenus cet effet au siège social.
}

\subparagraph{
  La cession des actions s'opère, l'égard des tiers et de la Société, par un ordre de mou compte à compte signé du cédant Ou dc son mandataire. Le mouvement cs mentlonné sur ces registres.
}

\subparagraph{
  13.2 actions ne sont négociables qu'après l'immatriculation de la Société au Registre du Commerce et des Sociétés en cas d'augmentation de capital, tes actions sont négociables compter de la réalisation définitive de celle-ci.
}

\subparagraph{
  13.3 Toute cession Œnctions de la Société au profit dun autre associé ou au profit d'un tiers est Soumise au respect du droit de préemption réservé aux associés, aux conditions Ci-après.
}

\subparagraph{
  Pour l'application des disposition du présent article:
}

\subparagraph{
  Le terme "actions" Sentcnd de toutes valeurs mobilières existantes ou futures, quelles qu'elles soient, y compris celles représentant ou susceptibles de représenter, meme à terme, le capital ou les droits de vote de la Société, 011 un droit sur répartition des bénéfices, ainsi que tous droits et bons permettant de devenir, par quelque moyen que ce soit, titulaire d'actions de la Société.
}

\subparagraph{
  Le terme "cession" signifie toute opération à titre onéreux ou gratuit entrainant le transfert de la pleine propriété, de la nue-propriété ou de l'usufnait des valeurs mobilières émises par la Société, savoir : cession, transmission, échnngc, apport cn société, fusion et opération assimilée, cession judiciaitc, constitution de trusts, nantissement, liquidation, transmission universelle de patrimoine. Le terme "Opératim de reclasxmcnt signifie totRe *ation de reclassement simple des actions de la SŒiété intervenant à de chacun des groupes crassociés, constitué par chXlue Société ct les sŒiétés Ou entités qtfelle contrôle directement ou indirectement au sens l'article 1.233-3
}

\subparagraph{
  13.4 Préemption
}

\subparagraph{
  L'associé qui envisage de céder ses actions doit informer par lettre recommandée avec accusé de réception, les autres associés de projet de cession indiquant :
  le nombre d'actions concernées ;\\
  l'identité du cessionnaire avec sa dénomination ;\\
  l'adresse de son siège social, le montant et la répartition de son capital ;\\
  l'identité des actionnaires et morales pour ces dernières la  compmitim dc leur capital, l'identité de leurs ainsi de suie tant que des personnes morales apparaitront ;\\
  l'identité dirigeants sociaux ;\\
  le prix les conditions de cession\\
}

\subparagraph{
  La date récg'ticm de la notification l'associé Cédant frit courir délai de deux (2) mois, à l'expiration duquel, si les droits préemptiM1 n'ont été exercés en totalité sur les actions le Cédant réaliser librement la cession projet. Sous réErve de la'ragrément à rarticlc des StautS.
}

\subparagraph{
  Chaque associé bénéficie d'un droit de pernption Str les Etions frisant IWet du projet de cession. Cc droit de préemption est exercé notificatim au Président dans les trente (30) jours au plus tard de la réception de la notification ci-dessus visée. Cette n'Mification est effectuée par lettre recommandée avec demande d'avis réceptim précisant le que chaque associé souhaite avoir.
}

\subparagraph{
  A l'expiration du délai trente (30) jours prévu à l'alinéa ci-dessus et avant celle du délai de deux mois ci-dessus, le Président thit notifier i l'associé Cédant pr lettre recommandée demande de réception les résultats de la préalptim.
}

\subparagraph{
  Si les droits de péemptim exercés s*ieurs au dont la cession envisagée. les actions concernées réparties sur le Président entre les associés qui ont notifié leur volonté au dc kur —iptim au capital de et dans la limite de leurs detnandcs.
}

\subparagraph{
  Si les droits rwécmption wnt infériars au nunbrc tions bat la envisagée, les droits de préemption réPJtés nivoir jamais été exercés et Cédant est libre réaliser la cession au profit cessionnaire mentionné dans notification. régrvc de resFcter la prœédure à rarticle Agréarnt  ci -apes.  En cas Œexercice du Œ0it préemption. la cession devra être dans un délai I S jours moyennant prix arntinné la rwtificatioa de rass«ié Caant.
}

\subparagraph{
  Si la cession porte Sur la majMité actions. elle donnera lieu la conclusion d'une convention de garantie (factif ct de Inssif, suivant usages cn la matière, basée sur unc situation cMnptablc de la la datc cessim, certifiée canmis.saire MIX de la Société.
}

\subparagraph{
  13.5 Agrément
}

\subparagraph{
  Les action ne peuvent cédées y qu'avec ragrément préalable de la collectivité des voix des asx»ciés diSFtsant du droit de vote. La mandc doit arc lettre recanmandée avec ŒaviS de réceflion adressée au Président de la Société et indiquant le nombre d'actions dont la cession est envisagée, le prix de la cession, les nom, prénoms, adresse, nationalité dc l'acquéreur ou s'il s'agit d'une personne morale, son l'identification complète (dénomination, siège social, numéro RCS, montant et répartition du capital, identité de ses dirigeants sociaux). Cette demande d'agrément est transmise par le Président aux associés.
}

\subparagraph{
  Le Président dispose d'un délai de trois (3) mois à compter de la réception de la demande d'agrément pour faire connaitre au Cédant la décision de la collectivité des associés. Cette notification est effectuée par lettre recommandée avec demande d'avis de réception. A défaut de réponse dans le délai ci-dessus, l'agrément sera réputé acquis.
}

\subparagraph{
  Les décisions dagrément ou de refus d'agrément ne sont pas motivées.
}

\subparagraph{
  En cas d'agrément, Passocié Cédant peut réaliser librement la cession aux conditions notifiées dans sa demande d'agrément. Le transfert des actions doit être réalisé au plus tard dans les trente jours de la décision d'agrément, à défaut de réalisation du transfert dans cc délai, l'agrément serait frappé de caducité.
}

\subparagraph{
  En cas de refus d'agrément, la Société est tenue dans un délai de un (l) mois à Compter de la notification du refus d'agrément, d'acquérir Ou de faire acquérir ICS actions de l'associé Cédant par un Ou plusieurs tiers agrŒs selon la procédure ci-dessus prévue.
  Si le rachat des actions n'est pas réalisé du fait de la Société dans ce délai d'un mois; l'agrément du ou des cessionnaires est réputé acquis.
  En cas d'acquisition des actions par la Société, celle-ci est tenue dans un délai de six (6) mois à compter de l'acquisition de les céder ou de les annuler.
  Le prix de rachat des actions par un tiers Ou par la Société est déterminé d'un commun accord entre les parties. A défaut d'accord, le prix sera déterminé à dire d'expert, dans les conditions de l'article 1843—4 du Code civil.
}

\subparagraph{
  13.6 Location
}

\subparagraph{
  La location des actions est interdite.
}

\section*{ARTICLE 14 - DROITS ET OBLIGATION ATTACHES AUX ACTIONS}


\subparagraph{
  14.1 Chaque action donne droit, dans tes bénéfices et l'actif social, à une part proportionnelle à la quotité du capital qu'elle représente et donne droit au vote ct la représentation dans les Assemblées Générales, dans les conditions fixées par les statuts.
}

\subparagraph{
  Tout associé a le droit d'être informé sur la marche de la Société ct dobtenir communication de certains documents sociaux aux époques et dans les conditions prévues par la loi et les statuts. Il bénéficie cn outre d'un droit d'information défini l'article 31 ci-après.
}

\subparagraph{
  14.2 Les associés ne supportent les pertes qu'à concurrence de leurs apports.
}

\subparagraph{
  Sous réserve des dispositions légales et statutaires, aucune majorité ne peut leur imposer une augmentation de leurs engagements. Les droits et obligations attachés à l'action suivent le titre dans quelque main qu'il passe.
}

\subparagraph{
  La possession d'une action comporte de plein droit adhésion aux décisions des associés et aux présents statuts. La cession comprend tous les dividendes échus et non payés et à échoir, ainsi éventuellement que la part dans ICS fonds dc réserve, sauf dispositions contraires notifiées à la Société.
}

\subparagraph{
  14.3 Chaque rois qu'il est nécessaire posséder un certain nombre actions exercer un droit quelconque, en cas déchange, de regroupement ou attribution de titres ou lors dune augmentation ou Œune réduction de capital, dune fusion ou de toute autre les associés un nombre dactions inférieur à celui requis. ne peuvent exercer ces droits qu'à la de faire leur affaire de roMention du nombre (factions requis.
}

\paragraph{
  TITRE III\\
  DIRECTION - FONCTIONNEMENT ET CONTRÔLE DE LA SOCIÉTÉ
}

\section*{ARTICLE 15 - PRESIDENT}

\subparagraph{
  La Société revésentée, adninistrée et dirigé par un président, physique ou morale, associé ou de la Société.
}

\subparagraph{
  Le Président est nommé renouvelé dans ses une durée de trois ans, par l'Assemblée Générale Ordinaire.
}

\subparagraph{
  Lorsquftlnc personne morale est nommée Président Ou dirigeant, les dirigeants de ladite morale sont soumis aux memes conditions et obligations et encourent les mêmes responsabilités civile ct vénale que s'ils étaient Président Ou dirigeant en leur nom propre, préjudice de la resBXIsabilité solidaire de la morale qtfils dirigent.
}

\section*{ARTICLE 16 - POUVOIRS DU PRESIDENT}

\subparagraph{
  Le Président assume. sous sa rewmsabilité. la Direction de la Société. Il la représente dans ses ramas avec les tiers, avec les les plus étendus. dans la limite de l'objet social Ct des pouvoirs de I  Asœrnblée Générale.
}

\subparagraph{
  Le Président ne peut engager la société pour des emprunts, garanties, rnrticipations ou cessions 'factifs sans rautorisation préalable de l'assemblée générale. plus, le ne Fut consentir caution, aval ou garantie, ns raccord préalable de
}

\subparagraph{
  Les décisions des associés limitant ses 1K'uvoirs inopposables aux tiers.
}

\subparagraph{
  Dans ses rapports avec tiers, le Président engage la Scxiété mène par les actes qui ne relèvent pas de l'objet social, moins qu'elle ne prouve que le tiers savait que l'acte dépassait cet Objet ou qu'il ne pouvait l'ignorer, compte tenu des circonstances, étant exclu quc la seule publication des statuts à constituer cette preuve.
}

\subparagraph{
  Le Président peut consentir à tout mandataire de son choix toutes délégations de pouvoirs qu'il nécessaires. dans la limite de ceux qui lui conférés par la loi ct les présents statuts.
}

\section*{ARTICLE 17 - REMUNERATION DU PRESIDENT}

\subparagraph{
  La fonction du Président ouvre droit à rémunération. elle est fixée les ass«xiés dans le dune collective exprirnée soit en assemblée Ou un acte sous seing privé.
}

\section*{ARTICLE 18 - CONVENTIONS EURE LA SOCIÉTÉ ET LE PRÉSIDENT}

\subparagraph{
  Les conventions qui peuvent être passées entre la Scxiété et son Président ou l'un de ses dirigeants sont Soumises aux formalités de contrôle prescrites par l'article L 227- I I du Code du commerce.
}

\subparagraph{
  Les conventions approuvées produisent néanmoins leurs effets, à charge la intérets et éventuellement pour le Président d'en tirer les conéquences dommageables pour la Société.
}

\subparagraph{
  Les dispositions qui pécèdent sont pas applicables aux conventions portant sur les *ations courantes et cmclues à des conditions normales.
}

\subparagraph{
  Les interdictions prévues à L 225—43 du Code du cornnErce s'appliquent, dans Jes conditions déterminées cet article, au président de la Sœiété.
}

\section*{ARTICLE 19 - COMMISSAIRES AUX COMPTES}

\subparagraph{
  S'il y a lieu et conformément à la Loi, les assœiés 'Esiglent un commissaire aux comptes pour autant que les conditions prévues à l'article L 227-9-1 du Code de commerce soient réunies.
}

\subparagraph{
  Un Ou plusieurs Commissaires aux Comptes titulaires nommés et exercent leur mission en conformité avec la Loi, ils ont pour mission permanente, à l'exclusion de toute immixtion dans la gestion, de vérifier les livres et les valeurs de la S'Eiété et de contrôler la régularité et la sincérité des comptes M'ciaux et d'en rendre cor* la collectivité des associés.
}

\paragraph{
  TITRE IV\\
  \\
  DÉCISIONS COLLECTIVES
}

\section*{ARTICLE 20 - FORME DES DECISIONS}

\paragraph{
  Les décisions des associés au choix du Président, prises en Assemblée Générale ou résultent du consentement des associés exprimé dans un acte seing privé. Elles peuvent également faire l'objet d'une consultation écrite.
}

\paragraph{
  Les Assembles Générales Ordinaires sont celtes qui appelées à prendre toutes décisions qui ne modifient pas ICS
}

\paragraph{
  Les Assemblées Générales Extraordinaires sont celles appelées à 'Eder ou autoriser des rn«xlifications directes ou des
}

\paragraph{
  Les délibérations des Assernblécs Générales obligent tous les associés. menc absents.
}

\section*{ARTICLE 21 - CONVOCATION ET DES ASSEMBLÉES}

\paragraph{
  Les Assemblées Générales sont convoquées soit par le Président, soit par un ou plusieurs associés représentant au moins 35 \% du capital social, soit un mandataire désigné par le Président du Tribunal de Commerce Statuant en référé i la demande ou plusieurs associés réunissant le tiers au moins du capital.
}

\paragraph{
  Elles peuvent également être convoquées par le Commissaire aux Comptes pour autant Commissaire aux comptes ait été désigné par les associés.
}

\paragraph{
  Pendant la période de liquidation, les Assemblées sont convoquées par le ou les liquidateurs. Les Assemblées Générales sont réunies au siège social ou en tout autre lieu indiqué dans ravis
}

\paragraph{
  Si tous les associés sont présents, la convocation peut être verbale.
}

\paragraph{
  La convocation est faite 15 jours avant la date de l'Assemblée soit par lettre simple ou recommandée adressée à chaque associé, soit par un avis inséré dans un Journal d'annonces légales du département du siège social, soit par télécopie, soit par courriel.
}

\paragraph{
  Lorsqu'une Assemblée n'a pu régulièrement délibérer, faute de réunir le quorum requis. La deuxième Assemblée et, le cas échéant, la deuxième Assemblée prorogée, sont convoquées dans les mêmes formes que la première et l'avis de convocation rappelle la date de la première et ru)roduit son ordre du jour.
}

\section*{ARTICLE 22 - ORDRE DU JOUR}

\paragraph{
  22.1 L'ordre du jour des Assemblées est établi par l'auteur de la convocation,
}

\paragraph{
  22.2 Un ou plusieurs associés, représentant au moins la quotité du capital social requise et agissant dans les conditions et délais fixés par la loi, ont la faculté de requérir, par lettre recommandée avec demande d'avis de réception, l'inscription à l'ordre du jour de rAssemblée de projets de résolutions.
}

\paragraph{
  22.3 L'Asscmblée ne peut délibérer sur une question qui n'est pas inscrite à l'ordre du jour, lequel ne peut être modifié sur deuxième convocation.
}

\section*{ARTICLE 23 - ADMISSION AUX ASSEMBLEES - POUVOIRS}

\paragraph{
  23.1 Tout a le droit de participer aux Assemblées Générales et aux personnellement ou par mandataire, quel que soit le nombre de ses actions, sur simple justification de son identité, dès lors que ses titres sont inscrits en compte son nom.
}

\paragraph{
  23.2 Un asscxié ne peut sc faire représenter que par un autre associé justifiant d'un mandat.
}

\section*{ARTICLE 24 - MODALITÉS DE TENUE DE L'ASSEMBLÉE - BUREAUX - PROCÈS-VERBAUX}

\paragraph{
  24.1 L'assemblée des Associés sc tient selon ICS modalités décrites dans les convocations.
}

\paragraph{
  Si la convocation est adressée par le Président, l'assemblée peut se tenir par tous moyens et toute façon, y compris par vidéo conférence, Skype, internet, ou autre moyen de communication et de transmission d'informations.
}

\paragraph{
  24-2 Chaque assemblée est présidée par IC Président et en son absence, par une spécialement désignée à Cet effet par l'assemblée.
}

\paragraph{
  Pour toute assemblée, dont la convocation n'a pas été faite par le président de la société. elle se tient obligatoirement au siège social de l'entreprise.
}

\paragraph{
  Pour toute assemblée, les assxiés émargent une feuille dc présence, émargement prend li forme d'un courriel ou d'un vœal enregistré pour toutes les assemblées non matérialisées par réunion assŒiés. Les courriels ou messages vocaux enregistrés transcrits annexés à la feuille tenue par le bureau.
}

\paragraph{
  Lors d'une réunion physique de les actionnaires, la feuille de vrésence est émargée par les Associés présents et les mandataires, à cette feuille présetre annexés les donnés à chaqtE mandataire.
}

\paragraph{
  La feuille de présence est exacte le burat' de rasEmblé.
}

\paragraph{
  Le de rassemblée est du préident absence une personne spéciakrnent désignée cet effet par rassemblée, d'un Secrétaire désigné par membres de l'assemblée, le Secrétaire étre dehœs des acti»nmires.
}

\paragraph{
  En cas de convocation que le Président exercice dans la société, rassemblée est présidée rur rauteur la convcxation.
}

\paragraph{
  A l'assemblée élit elle--aime un SErétaire qui peut étre pris en dehors de membres.
}

\paragraph{
  24.3 Les délibérations Assemblées des signés par le Président et le Secrétaire, ils sont consignés sur un registre spécial conformément à la loi. Les copies et extraits de Frœès-verbaux smt valablement certifiés dans l'un des deux.
}

\section*{ARTICLE 25 - QUOROM - VOTE}

\paragraph{
  25.1 Le quorum est calculé l'ensemble composant le capital social, le tout.  déduction faite actions prives du droit de vertu des de la loi ou des présents statuts.
}

\paragraph{
  25.2 Chaque action dontE droit Ime voix.
}

\paragraph{
  25-3. Le vote à main levée, ou aprzl nominal. au *ret, selon ce décide le bureau de l'Assemblée ou
}

\section*{ARTICLE 26 - ASSEMBLÉE GÉNÉRALE ORDNAIRE}

\paragraph{
  L'Assemblée Générale Ordinaire prexl qui pas de modifier les status.
}

\paragraph{
  L'Assernblée Générale Ordinaire est au moins tme fois ran. dans les six mois de la clôture de l'exercice social. statuer Sur les cet exercice, pus réserve de prolongation de ce délai décision de justice.
}

\paragraph{
  Elle ne valablement sur première si ou possèdent au moins Ic quart des actions ayant le droit de vote.
}

\paragraph{
  Aucun quorum n'est sur deuxième convtxatkm. Elle Statue à la majorité des voix dont disposent les associés présents ou représentés.
}

\section*{ARTICLE 27 - ASSEMBLÉE GÉNÉRALE EXTRAORDNAIRE}

\paragraph{
  L'Assemblée Générale Extraordimire IEut les stattÂs dans tounes leurs dispositions ct décider notamment la transfcmnation de la Smüté en d'une autre forme, civile ou Commerciale. Elle nè peut toutefois augmenter lès engageinene des sous opérations résultant d'un regroupement régulièrement effectué.
}

\paragraph{
  L'Assembléc Générale Extraordinaire ne peut délibérer valablement que si les associés présents ou représentés possèdent au moins. sur première convocation, le tiers et, sur deuxième convocation, le quad des actions ayant IC droit de Vote. A défaut de ce quorum, la deuxième Assemblée peut être prorogée à une date postérieure de deux mois au plus celle à laquelle elle avait été convoquée.
}

\paragraph{
  L'Assemblée Générale Extraordinaire statue la majorité des deux tiers des voix dont les associés présents ou représentés.
}

\paragraph{
  Toutefois, ne pourront être modifiées qu'à des associés, les clauses relatives à ragrément lors des cessions d'actions.
}

\paragraph{
  En outre, toutes décisions visant augmenter les engagements des associés ne peuvent être prises sans le de ceux-ci.
}

\section*{ARTICLE 28 - DROIT DE COMMUNICATION DES ASSOCIÉS}

\paragraph{
  Tout associé a le droit d'obtenir, avant toute consultation, communication des documents nécessaires pour lui permettre de se prononcer en connaissance de cause et de porter un jugement sur la gestion et le contrôle de la Société.
}

\paragraph{
  Chaque associé pourra se faire communiquer, sur simple demande, copie de tous documents comptables relatifs à ou à la situation financière de l'Entreprise.
}

\paragraph{
  TITRE V\\
  \\
  EXERCICE SOCIAL - COMPTES SOCIAUX - AFFECTATION ET RÉPARTITION DES BÉNÉFICES
}

\section*{ARTICLE 29 - EXERCICE SOCIAL}

\paragraph{
  L'exercice social commence le 1 janvier et se termine le 31 décembre de chaque année.
}

\paragraph{
  Le premier exercice social sera clos exceptionnellement le 31 décembre 2014.
}

\section*{ARTICLE 30 - INVENTAIRE - COMPTES ANNUELS}

\paragraph{
  Il est tenu une comptabilité régu lière des opérations conformément aux lois et du
}
\paragraph{
  À la clôture de chaque exercice, le Président dresse l'inventaire des divers éléments de l'actifet du passif Il dresse également les comptes annuels conformément aux dispositions du Titre Il du Livre I du Code de Commerce,
}
\paragraph{
  Il annexe au bilan un état des cautionnements, avals et garanties donnés par la Société et un état des sûretés consenties par elle,
}
\paragraph{
  Il établit un rapport de gestion contenant les indications fixées la loi.
}
\paragraph{
  Le rapport dc gestion inclut, le cas échéant, le rapport sur la gestion du groupe lorsque la Société doit établir et publier dcs comptes consolidés dans les conditions prévues par la loi.
}
\paragraph{
  ... Générale Extraordinaire remet de décir y a lieu i 12 Sceiété.-
}
\paragraph{
  Si la n'est IXOnoncée, le capital doit être. Sous résene des dispsitiOns légales relatives nu capital minimum et dans le délai fixé par là loi. réduit d'un montant égal i celui des Fttes qui n'ont pu être imputées Sur les réserves, si dans ce délai les capitaux pmxs nbnt PS été reconstitués à ctmcumence diJne valeur au moins égale la moitié du capital social.
}
\paragraph{
  Dans tous les cas, la "cision de l'Assembléc Générale doit formalités de requies dispositions réglementaires applicables.
}
\paragraph{
  En inobservatim de ces prescriptions. tout peut demander en justice la dissolution la Scxiété. Il en est de même si les asscxiés n'ont pu valablement.
}
\paragraph{
  Toutefris, le tribunal la si au jour il le la régularisaticm a lieu.
}

\section*{ARTICLE 34 - TRANSFORMATION}

\paragraph{
  La Société peut se transformer en Société d'une autre si, au moment de la transformation, elle a au moins deux ans (rexistence et Si elle a établi et fait approuver par les associés les bilans de ses deux premiers exercices.
}

\paragraph{
  La décision de transformation est prise sur le rapport d'un Commissaire aux Comptes à la transformation désigné par les associés ou à défaut Monsieur le Président du Tribunal de Commerce lequel doit attester que les capitaux propes au moins égaux au capital social.
}

\paragraph{
  La transformation en Société en Nom Collectif nécessite l'ucord de En c8. ks ci-dessus exigés.
}

\paragraph{
  La transformation en Société en Commandite Simple ou Etions les la des Statuts et avec raccord dc tous les associés devenant associés commandités.
}

\paragraph{
  La Responsabilité Limité est décidée dans prévues pour la modification des Statuts des Sociétés de cette forme,
}

\paragraph{
  La transformation qui entrainerait, l'augmentation des engagements des associés. soit la modification des clauses des présents Statuts exigeant l'unanimité des associés devra faire l'objet décision de ceux-ci.
}

\section*{ARTICLE 35 - DISSOLUTION - LIQUIDATION}

\paragraph{
  Hors les cas de dissolution prévus par la loi, et sauf prorogation régulière, la dissolution de la intervient l'expiration du ternie fixé Statuts Ou à la Alite de l'Assemblée Générale Extraordinaire des associés.
}
\paragraph{
  Un ou plusieurs liquidateurs alors nomnés par cette Assemblée Générale Extracydinaire conditions de quorum ct de muiorité prévues pour les Assemblées Générales Ordinaires.
}
\paragraph{
  Le liquidateur représente la smial est et le rssif acquitté liquidateur qui est investi des pouvoirs les plus étendus. Il répartit ensuite le solde disponible.
}
\paragraph{
  L'Assembléc associés luutoriser à affaires à en engager de nouvelles pour les besoins de la liquidation.
}
\paragraph{
  L'actif net Subsistant' après remboursement du nominal des actions est partagé toutes les actions.
}

\paragraph{
  TITRE VII\\
  \\
  CONTESTATIONS
}
\section*{ARTICLE 36 - CONTESTATIONS}

\paragraph{
  Tout différend entre les associés qui surviendrait dans l'interprétation de la validité ou l'exécution des statuts, que cc soit au cours de l'existence de la société ou après sa dissolution pendant le cours des actions de liquidation, seront de la compétence exclusive du Tribunal de Commerce de PARIS.
}

\paragraph{
  Toutefois afin de rcsvxcter l'esprit dans lequel les présents statuts ont été arrêtés, les parties s'engagent à mettre tout en œuvre pour régler amiablement leur différend.
}

\paragraph{
  pour se faire, devant une difficulté portant sur l'exécution ou l'interprétation des présentes, comme en cas de conflit entre elles, elles saisiront un médiateur inscrit au de l'Association des Médiateurs Européens et tenteront de parvenir un accord amiable,
}

\paragraph{
  Dans le cadre du protocole de médiation arrêté entre les parties en présence du médiateur, il sera porté que la durée de la médiation ne pourra pas être supérieure à trois (3) mois renouvelable une fois et ce conformément l'article 13 1-3 du Code de Procédure Civile.
}

\paragraph{
  A défaut d'accord amiable entre les parties Sur le choix d'un médiateur, passé un délai de trente (30) jours, à compter de la demande faite par une des parties aux autres associés, une lettre recommandée avec accusé dc réception de se rapprocher l'AME pour choisir un médiateur, la partie la plus diligente saisira Monsieur le Président du Tribunal de Commerce de PARIS Statuant en référé afin qu'il désigne un médiateur sous le Visa des articles 131 Ct suivants du Code de Procédure Civile.
}

\paragraph{
  TITRE VIII\\
  \\
  CONSTITUTION DE LA SOCIÉTÉ - DÉSIGNATION DES ORGANES SOCIAUX ACTES ACCOMPLIS POUR LA SOCIÉTÉ EN FORMATION
}
\section*{ARTICLE 37 - NOMINATION DU PRESIDENT}

\paragraph{
  En application de l'article L 225-16 du Code du le Président de la société est désigné pour une durée de 3 années qui expirera lors de l'assemblée Statuant sur les comptes du 3 exercice soit l'exercice qui sera Clos au 31 décembre 2016.
}

\paragraph{
  Le premier président est Madame Christine, Théree KIRCHNER née le 29 Février 1960 BEAUVAIS (60), de nationalité française, domiciliée au 52 rue de Courcelles 75008 PARIS.
}

\paragraph{
  Laquelle déclare accepter lesdites fonctions et satisfaire à toutes les conditions requises par la loi et les règlements pour son exercice.
}

\section*{ARTICLE 38 - Mandat de prendre compte des engagements pour le compte de la Société}

\paragraph{
  Les soussignés donnent mandat au présWent i toute Frsonm qu'il lui conviendra de se substituer à remet de prendre au nom et mr le compte de la Société tous actes et tous engagements entrant dans sœial et confomps à l'intérêt social permettant ou facilitant le commencement de l'exploitation commerciale de la smiété et nMamment :
}

\paragraph{
  Passati'tl de cmtraE avec les fournisseurs les de la et et encaissements afférents à ces ;
  paiement des loyers et charges de la société ;
  passation dc contrats leurs cmseils à la société ;
  Achat lœation des fournitures, matérEls et à rexploitatiul ;
  Engagement des frais de de la société
}

\paragraph{
  Les associés ratifient expressément et rgyenrœM au compte de la société tous ks engagements pris par les qui agi en nom avant la signature des présents statuts, Ou prendre par le président pour le rexploitation commerciale avant l'immatriculation au Registre du Commerce.
}

\paragraph{
  Les formalités de 'Riblieité precrites la loi et les r*nents effEtuées à la diligence du Président qui est *cialement mandaté pur signer tcMiS avis et autres formalités aux présents status.
}

\paragraph{
  Fait à\\
  L'an deux mille treize,\\
  Le
}

\paragraph{
  En six originaux dont :\\
  UN pour l'enregistrement,\\
  DEUX les délAts légaux.\\
  UN pour les archives sociales\\
  UN pour chacun des associés
}

\end{document}
