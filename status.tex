\documentclass[a4paper, 11pt]{article}
\usepackage[utf8]{inputenc}
\usepackage[T1]{fontenc}
\usepackage{lmodern}
\usepackage{graphicx}
% \usepackage[french]{babel}

\begin{document}

\title{« KORIUM »}
\author{
  Société par actions simplifiée au capital de 3.000 euros\\
  Siège social : 52, Rue de Courcelles\\
  75008 PARIS\\
}

\maketitle

\pagebreak

\paragraph{
  TITRE I\\
  FORME - DÉNOMINATION - OBJET - SIEGE - DURÉE
}

\section*{ARTICLE 1 - FORME}

\subparagraph{
  La Société est une Société par actions simplifiée régie par les dispositions  légales applicables et par les présents Statuts.
}

\subparagraph{
  Elle fonctionne indifféremment sous la même forme avec un ou plusieurs associés.
}

\subparagraph{
  Elle peut émettre toutes valeurs mobilières définies à l'article L 211-2 du Code monétaire et financier, donnant accès au capital ou de titres de créances, dans les conditions prévues par la loi et les présents statuts.
}

\section*{ARTICLE 2 - DÉNOMINATION}

\subparagraph{
  La dénomination sociale est « KORIUM »
}

\subparagraph{
  Dans tous les actes et documents émanant de la Société et destinés aux tiers, la dénomination doit être précédée ou suivie immédiatement des mots "Société par Actions Simplifiée" ou des initiales "S.A.S." et de l'énonciation du montant du capital social.
}

\section*{ARTICLE 3 - OBJET}

\subparagraph{
  La Société a pour Objet, directement ou indirectement, en FRANCE ou hors de FRANCE :
}

\subparagraph{
  - La vente, la distribution, IC négoce par tous moyens, tous supports et sous toutes formes en ce compris la vente en gros, demi-gros, au détail, au travers d'un réseau de distribution, par vente directe, que ce soit cn magasin ou par voie dématérialisée de tous produits fabriqués en cuir et/ ou en toutes autres matières et ce, en tout en partie.\\
  - Toutes prestations de services se rapportant à cette activité mais plus encore le Conseil sur l'organisation de tous types d'entreprise que ce soit dans le cadre de restructuration, de projet de développement, mais aussi l'acquisition, la conquêtc de parts de marché par des entreprises et en particulier tes structures artisanales,
}

\subparagraph{
  Et généralement toutes opérations financières, commerciales, industrielles, mobilières ou immobilières pour son compte ou pour celui d'autrui pouvant se rattacher directement ou indirectement aux Objets ci-dessus et susceptibles faciliter la réalisation ou le développement.
}
\section*{ARTICLE 4 - SIEGE SOCIAL}

\subparagraph{
  Le siège de la Société est sis :\\
  52, Rue de Courcelles\\
  75008 PARIS
}

\subparagraph{
  Il peut être transféré par décision du Président qui est habilité à modifier les statuts en conséquence.
}

\section*{ARTICLE 5 - DURÉE}

\subparagraph{
  La durée de la Société est de 99 années compter de son immatriculation au Registre du Commerce et des Sociétés, snuf les cas de prorogation ou dc dissolution anticipée.
}

\paragraph{
  Titre II\\
  APPORTS - CAPITAL - ACTIONS
}

\section*{ARTICLE 6 - APPORTS}

\section*{ARTICLE 7 - CAPITAL SOCIAL}

\subparagraph{
  Le capital social est fixé à la TROIS MILLE EUROS (3 000 EUROS).
}

\subparagraph{
  Il est divisé en trois cents (300) actions de dix (1O) euros chacune, souscrites en totalité et libérées en intégralité, les actions sont toutes de même catégorie.
}

\section*{ARTICLE 8 - AUGMENTATION DU CAPITAL SOCIAL}

\section*{ARTICLE 9 - LIBERATION DU CAPITAL}

\section*{ARTICLE 10 - REDUCTION DU CAPITAL, SOCIAL}

\section*{ARTICLE 11 - FORME DES ACTIONS}

\section*{ARTICLE 12 - INDIVISIBILITÉ DES ACTIONS}

\section*{ARTICLE 13 - CESSION ET TRANSMISSION DES ACTIONS}

\section*{ARTICLE 14 - DROITS ET OBLIGATION ATTACHES AUX ACTIONS}

\paragraph{
  TITRE III\\
  DIRECTION - FONCTIONNEMENT ET CONTRÔLE DE LA SOCIÉTÉ
}

\section*{ARTICLE 15 - PRESIDENT}

\section*{ARTICLE 16 - POUVOIRS DU PRESIDENT}

\section*{ARTICLE 17 - REMUNERATION DU PRESIDENT}

\section*{ARTICLE 18 - CONVENTIONS EURE LA SOCIÉTÉ ET LE PRÉSIDENT}

\section*{ARTICLE 19 - COMMISSAIRES AUX COMPTES}

\section*{ARTICLE 20 - FORME DES DECISIONS}

\section*{ARTICLE 21 - CONVOCATION ET DES ASSEWLÉES}

\section*{ARTICLE 22 - ORDRE DU JOUR}

\section*{ARTICLE 25 - QUOROM - VOTE}

\section*{ARTICLE 26 - ASSEMBLEE GENERALE ORDNAIRE}

\section*{ARTICLE 27 - GÉNÉRALE EXTRAORPJNA'RE}

\section*{ARTICLE 28 - COMMVNÇAT!ON}

\section*{ARTICLE 29 - EXERCICE SOCIAL}

\section*{ARTICLE 30 - INVENTAIRE - COMPTES ANNUELS}

\section*{ARTICLE 34 - TRANSFORMATION}

\section*{ARTICLE 35 - DISSOLUTION - LIQUIDATION}

\section*{ARTICLE 36 - CONTESTATIONS}

\section*{ARTICLE 37 - NOMINATION DU PRESIDENT}

\section*{ARTICLE 38 - Mandat de prendre compte des engagements pour le compte de la Société}

\end{document}


