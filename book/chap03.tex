\paragraph{
  L'ENIAC a été commandé en 1943 par l'armée américaine afin d'effectuer les calculs de balistique, c'est à dire toujours des calculs de trajectoire de missiles. Ce sont des calculs longs et fastidieux qui sont fait en moins d'une seconde par l'ordinateur au lieu des 10 minutes par un humain. La machine tombait souvent en panne mais elle était plutôt fiables pour l'époque. L'ENIAC pesait plus de 30 tonnes  et occupait 167 m2. Elle était composée de 20 calculateurs fonctionnant en parallèle et pouvait effectuer 100 000 additions ou 357 multiplications par seconde.
}

\paragraph{
  À partir de 1948 apparurent les premières machines à architecture de von Neumann. Contrairement aux machines précédentes, les logiciels et programmes ne sont pas stockés dans le même espace mémoire que les données. Tous les ordinateurs actuels sont de type von Neumann.
}

\paragraph{
  L'EDVAC : En 1945 Von Neumann fait un algorithme de tri pour l'EDVAC.
}
