\paragraph{
  Le processeur est le cerveau de l'ordinateur. On l'appelle aussi CPU, pour Central Processing Unit, qui se traduit par Unité Centrale de Traitement. C'est lui qui permet de manipuler des informations codées sous forme binaire, c'est à dire de Un et de Zéro. Et d'exécuter des instructions à partir de ses informations. Mais il ne permet pas de stocker des informations. Il est juste un éxécutant.
}

\paragraph{
  Le premier microprocesseur (Intel 4004) a été inventé en 1971. Il s'agissait d'une unité de calcul de 4 bits, cadencé à 108 kHz. Un microprocesseur regroupe la plupart des composants de calcul sur un seul circuit. Il réalisait 60 000 opérations par seconde. Depuis, la puissance des microprocesseurs augmente exponentiellement. Le 8080 sorti en avril 1974 est encore plus puissant.Quels sont donc ces petits morceaux de silicium qui dirigent nos ordinateurs ?
}

\paragraph{
  Ces plus petit microprocesseur vont permettre de réduire significativement la taille des computers et d'en faire un produit grand public dans les années 70. L'industrie de l'informatique va passer du BTB au BTC.
}

\paragraph{
  En janvier 1975 sort l'Altair 8800. Développé par des amateurs frustrés par la faible puissance et le peu de flexibilité des quelques ordinateurs en kit existant sur le marché à l'époque, de fut certainement le premier ordinateur personnel en kit produit en masse. Il était le premier ordinateur à utiliser un processeur Intel 8080.
}

\paragraph{
  En 1975 sortira aussi l'IBM 5100, machine totalement intégrée avec son clavier et son écran, qui se contente d'une prise de courant pour fonctionner.
}

\paragraph{
  Le Homebrew Computer Club est un club d'informatique de la Silicon Valley entre 1975 et 1986. Le club est initié par Gordon French et Fred Moore, qui s’étaient rencontrés à Menlo Park et souhaitent alors rendre l’informatique plus facilement accessible au grand public. L'Altair 8800 devint le sujet princpal de la première réunion du club. C'est lors de la présentation de l'Altaire que Wozniak eut l'idée d'assembler un ordinateur personnel grace au microprocesseur.
}

\paragraph{
  En janvier 1975, la revue Popular Mechanics faisait sa couverture avec l'Altair, le premier micor-ordinateur en kit. Pour 195 dollars on pouvait avoir un tas de composants à souder soi-même et fabriquer un computer. Mais une fois monté l'ordinateur ne pouvait pas faire grand chose avec. Bill Gates et Paul Allen lurent le magazine et se mirent à écrire un BASIC pour l'Altair. La machine retint aussi l'attention de Steve Jobs et Steve Wozniak.
}

\paragraph{
  Les microprocesseurs Intel étant trop cher à l'époque et Wozniak se rabat donc sur un Motorola 6800. Il écrit même le code source sur papier comme il ne peut pas se payer l'utilisation d'un ordinateur! Le 29 juin 1975 la machine est prête. "C'est la première fois dans l'histoire que quelqu'un tapait un caractère sur un clavier et le voyait s'afficher sur l'écran de son ordinateur, juste sous ses yeux".
}

\paragraph{
  L'Apple I se vendit à 666 dollars et environ 200 machines.
}

\paragraph{
  L'Apple II sort en 1977. Malgré son prix élevé (environ 1000 dollars) il prend l'avantage sur le TRS-80 et le Commodore PET lancé la même année. C'est un symbole d'ordinateur personnel à l'époque.
}

\paragraph{
  Steve Jobs et Bill Gates ne voulaient pas partager leur découvertes au *Homebrew Computer Club*. Une fois débarassé du partager leurs connaissances au club ils fondèrent leur société. Jobs convincut Wozniak de démissioner pour monter Apple. Et Gates a fondé Microsoft pour finaliser les ventes de BASIC pour l'Altair. Les libristes firent leur retour bien des années plus tard.
}

\paragraph{
  Steve Jobs pressentait un gros marché s'il arrivent à démocratiser ces machines, à les rendre plus intuitive, ergonomique. Et passer d'un marché BtoB à un marché BtoC avec des millions de nouveaux clients potentiels. Ils travaillaient donc d'arrache pied sur un système d'exploitation avec un environnement graphique au lieu d'un interprétateur de ligne de commandes.
}

\paragraph{
  Alan Kay chercheur au Xerox Parc travaille à l'époque sur des interfaces conviviales susceptibles de remplacer les lignes de commandes qui intimidaient tant le profane. Cette interface graphique était rendue possible par le bitmapping, soit l'affichage d'images matricielles. Chaque pixel est piloté point par point par la mémoire de l'ordinateur qui lui indique si le pixel est éteint ou allumé. L'image matricielle et les interfaces graphiques devinrent la caractéristiques des ordinateurs du PARC.
}

\paragraph{
  Xerox alors investisseur d'Apple autorisa Steve Jobs à rendre visite au centre de recherche situé à Palo Alto. À la différence des dirigeants de Xerox il sur tout de suite l'importance de ses travaux de recherche. S'agit-il d'un vol d'Apple ou d'une bourde de Xerox? Un peu des deux.
}
