\paragraph{
  Avant l'apparition des ordinateurs personnels il n'y avait que des grosses machines inutilisable pour le communs des mortels. On les appelait alors les *mainframe computer*. Les utilisateurs étaient surtout des chercheurs en intelligence artificiel, mathématiciens et quelques passionnées. Un des premiers ordinateur était le PDP 10. Dans les années 70, c'est sur cette ordinateur (*mainframe computer*) que  Richard Stallman (initiateur du mouvement du logiciel libre), Bill Gates et Paul Allen (les deux fondateurs de Microsoft) font leur premier pas en informatique. Steve Wozniak va lui commencé sur un ENIAC et un PDP-8 de Digital Equipment. Le constructeur principal des ces machines était IBM et sa machine le IBM 7094.
}

\paragraph{
  Le moyen d'intéragir avec ces grosses machines était,comme aujourd'hui, le système d'exploitation (Operating system ou OS). Et le plus populaires d'entre eux était UNIX. C'est un système d'exploitation (OS) créé en 1969 par Kenneth Thompson alors ingénieur au laboratoire Dell detenu par ATT. Il avait ensuite continué à être développé par les universités américaines. Il était compatible sur de nombreuses machines alors qu'à l'époque chaque ordinateur avait son propre système d'exploitation. Sur Unix, pour faire vite, vous donniez des instructions à un interprétateur de ligne de commande, un shell comme ci dessous. Le système d'exploitation se chargait de les exécuter. D'une manière générale, la quasi-totalité des PC ou mobile les plus courants (à l'exception des Windows) sont basés sur le noyau de Unix. Y compris ceux commercialisés par Apple. La taille des processeurs est immenses à l'époque, de plusieurs mètres cubes. Les ordniateurs tiennent dans plusieurs salles.
}

\paragraph{
  Au début des années 1960, les plus grosses sociétés d'informatiques sont IBM, Xerox, Dell, Commodore. Ce ne sont ques des constructeurs de machines physiques, du hardware. Pour elles, les OS et logiciels ne sont que la cerise sur la gateau. Ce sont les utilisateurs qui doivent écrire les logiciels. Etant donné que la plupart sont des chercheurs en Inteligence Artificielle, mathématiciens et quelques passionnées cela ne pose pas de problèmes.
}

\paragraph{
  Les circuits intégrés : Les premiers ordinateurs utilisant les circuits intégrés sont apparus en 1963. L'un des premiers usage a été dans les systèmes embarqués, notamment par la NASA dans l'ordinateur de guidage d'Apollo et par les militaires dans le missile balistique intercontinental LGM-30. Le circuit intégré permet le développement d'ordinateurs plus compacts.
}

\paragraph{
  En 1965, DEC lance le PDP-8.
}

\paragraph{
  Les générations de langages:
  1 codage machine direct en binaire
  2 langage assembleur
  3 langages évolués: Fortran, COBOL, Simula, APL etc.)
  4 langages structurés (Pascal, C++) et langage Objets
}

\paragraph{
  ARPANET (Source Wiki): ARPANET ou Arpanet (acronyme anglais de « Advanced Research Projects Agency Network », souvent typographié « ARPAnet »1) est le premier réseau à transfert de paquets développé aux États-Unis par la DARPA. Le projet fut lancé en 19662, mais ARPANET ne vit le jour qu'en 1969. Sa première démonstration officielle date d'octobre 1972.
}

\paragraph{
  Arpanet est créé afin d'unifier les techniques de connexion pour qu'un terminal informatique se raccorde à distance à des ordinateurs de constructeurs différents.
}

\paragraph{
  Le concept de commutation de paquets (packet switching), qui deviendra la base du transfert de données sur Internet, était alors balbutiant dans la communication des réseaux informatiques. Les communications étaient jusqu'alors basées sur la communication par circuits électroniques, telle que celle utilisée par le réseau de téléphone, où un circuit dédié est activé lors de la communication avec un poste du réseau.
}

\paragraph{
  Les ordinateurs utilisés étaient principalement des ordinateurs commerciaux de 3e génération construits par Digital Equipment Corporation (DEC), International Business Machines (IBM) ou Scientific Data Systems. Peut-être comprenaient-ils encore des Univac à tubes électroniques, technologie certes désuète en 1969 (où on abandonnait déjà les ordinateurs de deuxième génération transistorisés pour d'autres à circuits intégrés comme l'IBM 1130), mais c'est précisément pour cela que ces ordinateurs étaient libres pour un usage expérimental, les autres étant saturés de travaux3.
}

\paragraph{
  ARPANET a été écrit par le monde universitaire et non militaire, ce qui a probablement influencé l'Internet que l'on connait aujourd'hui4.
}

\paragraph{
  Le 29 Octobre 1969 le premier message était envoyé avec la machine ci-dessous sur un réseau appelé Arpanet entre les universités de Stanford et l'université de Californie à Los Angeles. C'est la naissance d'internet.
}
